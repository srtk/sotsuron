% !TEX root = thesis.tex
\chapter{おわりに}
\section{研究の成果}
%研究の成果というところがわずか4行しか述べられていません。
%少なくとも1ページは述べて下さい。

この研究では、深層学習をウェブ工学の問題に適用するにあたって、その特徴である高い精度を落とすことなく、出来るだけ簡便に応用するための方法論とノウハウを調査した。\par
まず、深層学習のアルゴリズムを実装する際に、考えられる問題点として、「分類精度の再現の問題」、「実装難易度の問題」、「学習時間の問題」の3つを挙げた。次に、この3つを解決ないし緩和するための対策を考えた。\par
ハードウェア面ではGPUによる高速演算の力を利用することで、「学習時間の問題」を大きく解決することが出来た。またソフトウェア面では、適切な深層学習のライブラリを選ぶことが、「実装難易度の問題」「学習時間の問題」に対する有効な対策であることを示した。\par
ここで、現段階で使える深層学習のライブラリについて、その性能を比較した。その結果、Pylearn2とDeep Learning Tutorialというライブラリが、2つの問題を解決する上で適当であることがわかった。\par
さらに、残った「分類精度の再現の問題」をこの2つのライブラリで解決できるかどうか、それぞれのライブラリに実装されている深層学習のモデルにより、実際のデータセットに対する分類精度の再現実験を行った。結果として、Pylearn2に実装されているMaxout Networkというモデルを用いると、Maxout Networkを発表した論文にて書かれている精度をほぼ再現することが出来ることがわかった。また、この実験ではGPUによる学習時間短縮の効果も確かめることができた。\par
結論として、現段階で深層学習をウェブ工学の問題に適用するためには、Pylearn2にあるMaxout Networkというモデルを使って、GPUを搭載したマシンの上で分類プログラムを実行すればよいことがわかった。

\section{深層学習技術の将来的な展望}
%Deep Learning技術の将来的な展望であれば、7章に本研究の意義を再度示すような目的で、書けばよいです。
現在の深層学習が比較的得意とする画像データ処理について、単純な認識タスクで大量の訓練データが用意されている条件ならば、畳み込みレイヤーによる特徴抽出が非常に良い性能を示す。しかし、問題設定を少し変えて、「シンボルを1つだけ見せて、他のシンボルの中から同じものを選ばせる(one-shot classification)」「シンボルを1つだけ見せて、同じシンボルを機械と人間に書かせ、どちらが上手く出来るか競う(one-shot generation)」といった、データ数が非常に少ない条件下では、まだ人間が手間をかけて作った素性を使った方が性能が高いことがある\cite{lake2013one-shot}。画像をただ認識するだけでなく、良い素性の作り方を少ないデータから学習できるようになれば、繰り返し回数を減らし学習時間を短縮することにもつながり、応用の幅が広がると思われる。\par
現在の深層学習では、画像のフィルタで何が学習されているのかを、部分的に可視化することはできる。しかし、文書解析において、どのような表現を学習したのかを、人間に理解できる形でみることは難しい。言い換えれば、学習によって、分類器がどこに注目するようになったのか、文構造や、感情、文体などに対応するニューロンが存在しているのか、といった情報が、人間に理解できる形になっていない。この部分をどうやって可視化して、人間に理解できる状態にするか、そこからどのような知見を得られるのか、あるいはそもそも人間には理解出来る知識として取り出せるのかどうか、といったことを調べることにより、表現学習としての深層学習の側面を、さらに活かすことができると思われる。例えば、深層学習にょって学習された、データの着目点や抽象化のポイントを、人間が真似することによって、人間の方が機械から知識を習得し、さらに発展させることも考えられる。
