% !TEX root = thesis.tex
\chapter{関連研究}
\section{Web工学と機械学習}
この節では、Web工学における課題をいくつか例示し、それらの課題を解決するために、機械学習がどのように用いられてきたかを概観する。
\subsection{Recommendation System}
主にWebショッピングを含むサイトや、Web広告の配信において求められる技術である。Webサイトを閲覧しているユーザに対し、そのユーザが購入したいと商品を予測して、Web上の広告などの形で推薦する。適切な広告を表示することにより、ユーザの購買行動を促進することが出来る。
Recommendation Systemの研究は、1992年のTapestryシステムに始まる\cite{goldberg}。また、少なくとも90年代の終わりには、Amazon.com, CDNOW, eBay, Levis, GroupLensなど、様々なサイトにてRecommendation Systemは利用されていた\cite{Resnick:1997:RS:245108.245121,Schafer:1999:RSE:336992.337035}。
Recommendation Systemを実現するための、機械学習のテクニックとしては、大きく2種類が挙げられる\cite{Koren:2009:MFT:1608565.1608614}。1つは、Content Filteringと呼ばれ、ユーザや商品の属性や購買傾向を学習することことで、推薦を行う。もう1つはCollaboraitive Filteringと呼ばれており、ユーザや商品の属性を扱う代わりに、購入や評価といった、ユーザの過去の行動を基にして、推薦を行う。
\subsection{Link Prediction}
グラフにおいて、現在のノード間接続から未来の接続を予測する
ソーシャルグラフの予測、タンパク質の反応予測(PPI)に有効である
[加筆]
\subsection{Sentiment Analysis}
ユーザの感想データを手に入れられるようになってきた
事実だけでなく、他の人がどのような感情を抱いているのかを分析したい
opinion miningという語も広義には似た内容を指している
2001-2003に研究が出始めた
common neighbors
Jaccard係数 SVDによる次元削減
Stanfordによるデモに言及
[加筆]
\subsection{Learning to Rank}
多様なランキング素因を組み合わせて、ランキング関数を作成
→どの組み合わせが有効か、機械学習する
[加筆]
\section{機械学習で利用される、代表的な分類器}
機械学習のプロセスは、「入力データを、数学的モデルで使える素性に変換する」「素性を数学的モデルに入力して、出力値を得る」「出力を見ながら、モデルを修正する」という行程に大きく分けられる。データの分類問題を機械学習で解く場合、モデルによる出力値が分類結果に対応するよう、モデルを学習させることになる。この場合、モデルのことを分類器とも呼ぶ。
機械学習において、素性への変換部分は、データの種類に大きく依存する。一方、分類器に用いる数学的モデルと、モデルの改修法、つまり学習法は、汎用的に使うことができる。あるいは、画像や音声、文章といったデータの多様性を、素性という一般的な数値に落とし込むことで吸収して、汎用的分類モデルでも学習できるようにしている。
Deep Learning、あるいはDeep Neural Network(多層ニューラルネットワーク)は、汎用的分類モデルの一種である。ここでは、Deep Learningの他にどのような分類器が存在するのか、代表的なものを述べる。
\subsection{線形識別モデル}

\subsection{ロジスティック回帰}

\subsection{パーセプトロン}
線形関数しか近似できないことがわかり下火になった

\subsection{ニューラルネットワーク}
ニューラルネットワークは、人間の脳の構造を模倣した数学的モデルである。人間の脳は、ニューロンと呼ばれる神経細胞が大量に接続されて出来ている。ニューロンが電気信号を伝達することで、様々な脳の働きが行われている、と考えられている。[模式図など加筆]\\
・隠れ層の追加により、任意の非線形関数を近似可能\\
・バックプロパゲーションの弱点(sigmoidが反応しなくなる)

\subsection{Restricted Boltzman Machine}

\subsection{Support Vector Machine}
Support Vector Macine(SVM)は、データを2つのクラスに分類する能力を持っている。
カーネルマジックとマージン最大化により、優れた精度を出している。SVMは、画像認識などに既に実用化されている。[加筆]\\
SVMは、広くその信頼性が認められたモデルの1つであり、ライブラリの利用方法も確立している。\\
SVMを簡単に使えるようにlibsvmやliblinearというライブラリが存在している。これらのライブラリを使うと、簡単な所定の方式に沿って入力データファイルを用意し、CUI上で2,3回の操作をするだけで、SVMによる分類を行わせることができるライブラリである。このとき、利用者が自分でプログラムを書く必要は全くない。プログラムを書かないで済むと、手軽に利用することができ、またバグを起こす危険性が非常に少なく安全に使うことができる。\\
Deep Learningについては、このようなライブラリはまだ存在していないため、Deep Learningの代表的なアルゴリズムについて、プログラム無しで利用できるようなライブラリの整備が望まれる。
