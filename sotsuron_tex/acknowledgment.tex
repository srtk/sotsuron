% !TEX root = thesis.tex
\chapter*{謝辞}
\addcontentsline{toc}{chapter}{謝辞}
本研究の執筆にあたり、指導教官である東京大学大学院工学系研究科総合研究機構准教授である松尾豊先生は、研究テーマの方向性を決める初期段階から何度も相談に乗って頂き、ご多忙の中私の曖昧な質問にも快くアドバイスを返して下さいました。また、調査・実験段階でも私が研究の方向性を見失わないよう助言をして下さった他、執筆段階では論文の書き方の基本的な心構えから教えてくださり、私の拙い草稿に何度も目を通して、修正のコメントをして下さいました。\par
東京大学知の構造化センター特認講師の中山浩太郎先生は、深層学習の実装段階において、ハードウェアとソフトウェア両面の多彩な知識によって、深層学習の研究を支えてくださり、また様々な補助ライブラリを提供して下さいました。東京大学情報理工学系研究科創造情報学専攻講師の中山英樹先生からは、画像認識研究の基本的なポイントについて、貴重な示唆を頂きました。\par
東京大学松尾研究室博士課程の大澤さんは、研究生活の全行程に渡って、何をすれば良いのかゼロから手取り足取り教えてくださいました。特に、執筆段階では、私がどのような初歩的な質問をしても、懇切丁寧に教えて下さり、執筆の大きな助けを頂きました。同研究室博士課程の榊さん、沼さん、関さんにも、研究の基本やテーマの方向性、スケジュールの考え方や執筆ソフトの扱いなどについて、様々なことを教えて頂きました。同研究室修士課程の飯塚さんや、同研究員の那須野さんには、ご自身も研究でお忙しい中、度重なる私のサーバ関連の報告に対して、長時間の対応をして下さり、非常に助かりました。同研究室学部4年生の川上さん、キムさん、安野さんには、長い卒業論文執筆期間の中で何度も励まされ、また時には切磋琢磨する相手としての緊張感をももたらして下さいました。他研究室のメンバーの方々にも、研究会を始めとして様々な場面で助言や励ましの言葉を頂きました。\par
本研究は、様々な方の協力が無くては完成し得ないものでした。ここにお礼を申し上げます。ありがとうございました。

\begin{flushright}
東京大学工学部システム創成学科 \\
知能社会システムコース\\
松尾研究室 4年 黒滝 紘生\\
平成26年2月6日\\
\end{flushright}