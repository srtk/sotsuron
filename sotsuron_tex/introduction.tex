% !TEX root = thesis.tex
\chapter*{概要}
\addcontentsline{toc}{chapter}{概要}
近年機械学習の分野において、深層学習と呼ばれるアルゴリズム群が優れた成果を納めている。Web工学でも、深層学習を応用することによる発展が期待される。\par
しかし、深層学習は歴史の浅い発展途上の技術であり、どのアルゴリズムを定番とすれば良いのか、試行錯誤の段階にある。これは、各アルゴリズムの改良点が次々と見つかっていることに加え、各アルゴリズムの得手不得手や、学習性能が高くなる原理の詳細など、解明されていない部分が多いことが、主な原因である。\par
また、現在の深層学習技術では、他のアルゴリズムに比べて学習にかかる時間が長いことが多く、ハードウェア性能が低いマシンでは、アルゴリズムを実用的な時間で実行すること自体が容易ではない。実行時間の長さをカバーするため、GPUを用いて演算をスピードアップさせる手法が確立されつつあるが、まだ広く一般に浸透した技法にまでは至っていないため、参入障壁の一つになっている。\par
このような原因により、深層学習技術に関心を持っていても、まず実際の問題に深層学習を試行すること自体が困難であり、Web工学など応用分野に深層学習を適用する場合のハードルは更に高くなっている。特に、国内での研究開発は遅れており、早急なキャッチアップが必要である。\par
このような現状を踏まえ、本研究では、Web工学における応用を見据えつつ、深層学習を様々な問題に応用するための方法論を整理した。具体的には、現在公開されている深層学習な目標としては、深層学習の特徴である高い学習性能や分類精度を確実に利用できて、その上で出来る限り、実行時間の短さ、実行プログラムの使いやすさ、アルゴリズムの調整・改良の容易さを兼ね備えた方法を確立する。

\begin{comment}このような原因により、深層学習技術に関心を持っていても、まず実際の問題に深層学習を試行すること自体が困難であり、応用技術開発のハードルは更に高くなっている。アルゴリズムが開発途上で確定できていないため、公開されているライブラリも、現状では、開発用途や実験的なものが多くなってしまっている。そもそも有力なアルゴリズムに対応する実装が用意されていない場合や、問題に応じて自らアルゴリズムの細部を調整しなければならない場合もある。ライブラリがGPU専用に書かれていることが徒となり、GPUを持っていないと実行自体ができなくなることも考えられる。例え実行するところまで到達できたとしても、提供されているのがアルゴリズムのダイジェスト版でしかなく、論文にて示されている精度を手元で再現することが出来ない場合も多い。標準と言える公開ライブラリが確立していない状況なので、Web工学など応用分野に深層学習を適用したいと考えても、プログラム開発に長い時間がかかってしまい、開発における大きな障壁となっている。特に、国内での研究開発は遅れており、早急なキャッチアップが必要である。
\end{comment}
