% !TEX root = thesis.tex
\chapter*{概要}
\addcontentsline{toc}{chapter}{概要}
近年機械学習の分野において、深層学習(Deep Learning)と呼ばれるアルゴリズム群が優れた成果を納めている。ウェブ工学でも、深層学習を応用することによる発展が期待される。\par
しかし、深層学習は歴史の浅い発展途上の技術であり、どのアルゴリズムを定番とすれば良いのか、試行錯誤の段階にある。これは、各アルゴリズムの改良点が次々と見つかっていることに加え、各アルゴリズムの得手不得手や、学習性能が高くなる原理の詳細など、解明されていない部分が多いことが、主な原因である。ライブラリの整備は十分に進んでおらず、まず実際の問題に深層学習を実装し、期待している精度を再現すること自体が困難である。長い学習時間も開発のネックとなり、深層学習技術に関心を持っていても、開発・応用を始めるためのハードルは高くなっている。特に、国内での研究開発は遅れており、早急なキャッチアップが必要である。\par
そこで本研究では、ウェブ工学における応用を見据えつつ、深層学習を様々な問題に応用するための方法論を整理した。現在の深層学習技術において問題となっている、「分類精度再現の難しさ」「実装難易度の高さ」「実行時間の長さ」を解決するため、現在公開されているソースコードを比較し、どの実装やモデルを用いれば、深層学習に期待される精度を手元でも再現できるのかを調査した。また、実証実験を行い、分類精度の再現性についてチェックを行った。この実験の結果を受け、深層学習の技術を応用して、精度の高い機械学習を行うための方法論について、提言を行った。この際、実行時間の長さの問題、実行プログラムの使いやすさ、アルゴリズムの調整・改良の容易さなども視野に入れ、実際の問題において深層学習が応用しやすくなるよう考慮した。